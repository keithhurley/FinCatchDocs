% Options for packages loaded elsewhere
\PassOptionsToPackage{unicode}{hyperref}
\PassOptionsToPackage{hyphens}{url}
\PassOptionsToPackage{dvipsnames,svgnames,x11names}{xcolor}
%
\documentclass[
  letterpaper,
  DIV=11,
  numbers=noendperiod]{scrreprt}

\usepackage{amsmath,amssymb}
\usepackage{lmodern}
\usepackage{iftex}
\ifPDFTeX
  \usepackage[T1]{fontenc}
  \usepackage[utf8]{inputenc}
  \usepackage{textcomp} % provide euro and other symbols
\else % if luatex or xetex
  \usepackage{unicode-math}
  \defaultfontfeatures{Scale=MatchLowercase}
  \defaultfontfeatures[\rmfamily]{Ligatures=TeX,Scale=1}
\fi
% Use upquote if available, for straight quotes in verbatim environments
\IfFileExists{upquote.sty}{\usepackage{upquote}}{}
\IfFileExists{microtype.sty}{% use microtype if available
  \usepackage[]{microtype}
  \UseMicrotypeSet[protrusion]{basicmath} % disable protrusion for tt fonts
}{}
\makeatletter
\@ifundefined{KOMAClassName}{% if non-KOMA class
  \IfFileExists{parskip.sty}{%
    \usepackage{parskip}
  }{% else
    \setlength{\parindent}{0pt}
    \setlength{\parskip}{6pt plus 2pt minus 1pt}}
}{% if KOMA class
  \KOMAoptions{parskip=half}}
\makeatother
\usepackage{xcolor}
\setlength{\emergencystretch}{3em} % prevent overfull lines
\setcounter{secnumdepth}{5}
% Make \paragraph and \subparagraph free-standing
\ifx\paragraph\undefined\else
  \let\oldparagraph\paragraph
  \renewcommand{\paragraph}[1]{\oldparagraph{#1}\mbox{}}
\fi
\ifx\subparagraph\undefined\else
  \let\oldsubparagraph\subparagraph
  \renewcommand{\subparagraph}[1]{\oldsubparagraph{#1}\mbox{}}
\fi


\providecommand{\tightlist}{%
  \setlength{\itemsep}{0pt}\setlength{\parskip}{0pt}}\usepackage{longtable,booktabs,array}
\usepackage{calc} % for calculating minipage widths
% Correct order of tables after \paragraph or \subparagraph
\usepackage{etoolbox}
\makeatletter
\patchcmd\longtable{\par}{\if@noskipsec\mbox{}\fi\par}{}{}
\makeatother
% Allow footnotes in longtable head/foot
\IfFileExists{footnotehyper.sty}{\usepackage{footnotehyper}}{\usepackage{footnote}}
\makesavenoteenv{longtable}
\usepackage{graphicx}
\makeatletter
\def\maxwidth{\ifdim\Gin@nat@width>\linewidth\linewidth\else\Gin@nat@width\fi}
\def\maxheight{\ifdim\Gin@nat@height>\textheight\textheight\else\Gin@nat@height\fi}
\makeatother
% Scale images if necessary, so that they will not overflow the page
% margins by default, and it is still possible to overwrite the defaults
% using explicit options in \includegraphics[width, height, ...]{}
\setkeys{Gin}{width=\maxwidth,height=\maxheight,keepaspectratio}
% Set default figure placement to htbp
\makeatletter
\def\fps@figure{htbp}
\makeatother
\newlength{\cslhangindent}
\setlength{\cslhangindent}{1.5em}
\newlength{\csllabelwidth}
\setlength{\csllabelwidth}{3em}
\newlength{\cslentryspacingunit} % times entry-spacing
\setlength{\cslentryspacingunit}{\parskip}
\newenvironment{CSLReferences}[2] % #1 hanging-ident, #2 entry spacing
 {% don't indent paragraphs
  \setlength{\parindent}{0pt}
  % turn on hanging indent if param 1 is 1
  \ifodd #1
  \let\oldpar\par
  \def\par{\hangindent=\cslhangindent\oldpar}
  \fi
  % set entry spacing
  \setlength{\parskip}{#2\cslentryspacingunit}
 }%
 {}
\usepackage{calc}
\newcommand{\CSLBlock}[1]{#1\hfill\break}
\newcommand{\CSLLeftMargin}[1]{\parbox[t]{\csllabelwidth}{#1}}
\newcommand{\CSLRightInline}[1]{\parbox[t]{\linewidth - \csllabelwidth}{#1}\break}
\newcommand{\CSLIndent}[1]{\hspace{\cslhangindent}#1}

\KOMAoption{captions}{tableheading}
\makeatletter
\makeatother
\makeatletter
\@ifpackageloaded{bookmark}{}{\usepackage{bookmark}}
\makeatother
\makeatletter
\@ifpackageloaded{caption}{}{\usepackage{caption}}
\AtBeginDocument{%
\ifdefined\contentsname
  \renewcommand*\contentsname{Table of contents}
\else
  \newcommand\contentsname{Table of contents}
\fi
\ifdefined\listfigurename
  \renewcommand*\listfigurename{List of Figures}
\else
  \newcommand\listfigurename{List of Figures}
\fi
\ifdefined\listtablename
  \renewcommand*\listtablename{List of Tables}
\else
  \newcommand\listtablename{List of Tables}
\fi
\ifdefined\figurename
  \renewcommand*\figurename{Figure}
\else
  \newcommand\figurename{Figure}
\fi
\ifdefined\tablename
  \renewcommand*\tablename{Table}
\else
  \newcommand\tablename{Table}
\fi
}
\@ifpackageloaded{float}{}{\usepackage{float}}
\floatstyle{ruled}
\@ifundefined{c@chapter}{\newfloat{codelisting}{h}{lop}}{\newfloat{codelisting}{h}{lop}[chapter]}
\floatname{codelisting}{Listing}
\newcommand*\listoflistings{\listof{codelisting}{List of Listings}}
\makeatother
\makeatletter
\@ifpackageloaded{caption}{}{\usepackage{caption}}
\@ifpackageloaded{subcaption}{}{\usepackage{subcaption}}
\makeatother
\makeatletter
\@ifpackageloaded{tcolorbox}{}{\usepackage[many]{tcolorbox}}
\makeatother
\makeatletter
\@ifundefined{shadecolor}{\definecolor{shadecolor}{rgb}{.97, .97, .97}}
\makeatother
\makeatletter
\makeatother
\ifLuaTeX
  \usepackage{selnolig}  % disable illegal ligatures
\fi
\IfFileExists{bookmark.sty}{\usepackage{bookmark}}{\usepackage{hyperref}}
\IfFileExists{xurl.sty}{\usepackage{xurl}}{} % add URL line breaks if available
\urlstyle{same} % disable monospaced font for URLs
\hypersetup{
  pdftitle={FinCatch Documentation},
  pdfauthor={Keith Hurley et al},
  colorlinks=true,
  linkcolor={blue},
  filecolor={Maroon},
  citecolor={Blue},
  urlcolor={Blue},
  pdfcreator={LaTeX via pandoc}}

\title{FinCatch Documentation}
\author{Keith Hurley et al}
\date{\texttt{r\ format(Sys.time(),\ \textquotesingle{}\%d\ \%B,\ \%Y\textquotesingle{})}}

\begin{document}
\maketitle
\ifdefined\Shaded\renewenvironment{Shaded}{\begin{tcolorbox}[breakable, boxrule=0pt, sharp corners, enhanced, borderline west={3pt}{0pt}{shadecolor}, frame hidden, interior hidden]}{\end{tcolorbox}}\fi

\renewcommand*\contentsname{Table of contents}
{
\hypersetup{linkcolor=}
\setcounter{tocdepth}{2}
\tableofcontents
}
\bookmarksetup{startatroot}

\hypertarget{Home}{%
\chapter*{Home}\label{Home}}
\addcontentsline{toc}{chapter}{Home}

This is documentation for the FinCatch Data System of the Nebraska Game
and Parks Fishery Division. FinCatch stores and provides analysis of
standard fisheries population surveys. This set of documentation is an
accumulation of both developmental and instructional documentation.

\part{System Components}

The FinCatch data system is comprised of a number of separate
components, including:

\hypertarget{fincatch}{%
\subsection*{FinCatch}\label{fincatch}}
\addcontentsline{toc}{subsection}{FinCatch}

FinCatch is the central website that provides data management
capabilities as well as links to other components. FinCatch is written
in the asp.mvc framework of .NET 6.

\hypertarget{fincatch-database}{%
\subsection*{FinCatch Database}\label{fincatch-database}}
\addcontentsline{toc}{subsection}{FinCatch Database}

The backend database for FinCatch is built in Microsoft SQL Server.

\hypertarget{fincatchde}{%
\subsection*{FinCatchDE}\label{fincatchde}}
\addcontentsline{toc}{subsection}{FinCatchDE}

\hypertarget{fincatchag}{%
\subsection*{FinCatchAG}\label{fincatchag}}
\addcontentsline{toc}{subsection}{FinCatchAG}

\hypertarget{fincatchra}{%
\subsection*{FinCatchRA}\label{fincatchra}}
\addcontentsline{toc}{subsection}{FinCatchRA}

\hypertarget{fincatchanalysis-r-package}{%
\subsection*{FinCatchAnalysis R
Package}\label{fincatchanalysis-r-package}}
\addcontentsline{toc}{subsection}{FinCatchAnalysis R Package}

\hypertarget{fincatchaccess-r-package}{%
\subsection*{FinCatchAccess R Package}\label{fincatchaccess-r-package}}
\addcontentsline{toc}{subsection}{FinCatchAccess R Package}

\hypertarget{fincatchwebapi}{%
\subsection*{FinCatchWebApi}\label{fincatchwebapi}}
\addcontentsline{toc}{subsection}{FinCatchWebApi}

\part{Analysis R Package}

\hypertarget{overview}{%
\section*{Overview}\label{overview}}
\addcontentsline{toc}{section}{Overview}

The FinCatchAnalysis (FCA) R package centralizes standard analysis
functions for the FinCatch system and promotes DRY and reusable analysis
code practices. The FCA package is built on top of R6 classes which
provides a standard programming interface for users of the R package.
Analysis functions are available for each individual analysis available
and results from each analysis function returns results encapsulated in
an R6 class object. All public functions in the package are prefixed
with ``fca\_''. Package R6 objects are prefixed with ``fco\_''.

\hypertarget{analysis-functions}{%
\section*{Analysis Functions}\label{analysis-functions}}
\addcontentsline{toc}{section}{Analysis Functions}

Each analysis function in the FCA package provides a single call for an
independent analysis and returns a function specific R6 object built on
the base fco\_ object that provides standard methods and data objects.

\hypertarget{analysis-return-objects}{%
\section*{Analysis Return Objects}\label{analysis-return-objects}}
\addcontentsline{toc}{section}{Analysis Return Objects}

\hypertarget{fincatchanalysis-r-package-1}{%
\chapter{FinCatchAnalysis R
Package}\label{fincatchanalysis-r-package-1}}

\hypertarget{overview-1}{%
\section*{Overview}\label{overview-1}}
\addcontentsline{toc}{section}{Overview}

The FinCatchAnalysis (FCA) R package centralizes standard analysis
functions for the FinCatch system and promotes DRY and reusable analysis
code practices. The FCA package is built on top of R6 classes which
provides a standard programming interface for users of the R package.
All public functions in the package are prefixed with ``fca\_''. Package
R6 objects are prefixed with ``fco\_''.

\hypertarget{analysis-functions-1}{%
\section{Analysis Functions}\label{analysis-functions-1}}

Each analysis function in the FCA package provides a single call for an
independent analysis and returns a function specific R6 object built on
the base fco\_ object that provides standard methods and data objects.

\hypertarget{analysis-return-objects-1}{%
\section{Analysis Return Objects}\label{analysis-return-objects-1}}

\part{References}

\hypertarget{refs}{}
\begin{CSLReferences}{0}{0}
\end{CSLReferences}



\end{document}
